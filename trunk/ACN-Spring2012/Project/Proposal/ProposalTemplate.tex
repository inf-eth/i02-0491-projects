% 
% This is a latex file showing you what I expect in your proposals.
% Your proposal should be from 2 to 3 pages long.  
% 
% Just save everthing below starting with the \documentclass line into a
% file and follow the directions in the comments following the
% \documentclass line.
% 
% You do not have to write your proposal and report in latex, but this
% is a good opportunity to learn it if you have never used it!
% 
% 						Chuck Anderson

\documentclass{article}
% Set margins to 1 inch
\setlength{\oddsidemargin}{0in}
\setlength{\evensidemargin}{0in}
\setlength{\headheight}{12pt}
\setlength{\headsep}{42pt}
\setlength{\topmargin}{-54pt}
\setlength{\textwidth}{6.5in}
\setlength{\textheight}{9in}
\usepackage{graphicx}

%Comments start with %
% To process this file, produce a poscript version, and view it, do
%  latex proposal
%  dvips -o proposal.ps proposal.dvi
%  gv proposal.ps
% You can also convert this to an Adobe Acrobat file by doing
%  distill proposal.ps
% This produces proposal.pdf

\begin{document}

\title{The Best Semester Project Ever}
\author{Me N. You}
\maketitle

\section{What}

Explain what you are going to be reading about or implementing for
your semester project.  Say why this topic interests you.  Convince me
that this topic is exciting to you and that you will learn a lot from
working on it.  Specify exactly what your goal is for the project, in
terms of what you want to learn from doing this.

Your proposal should be at least three pages long.  You may use LaTex
or another formatter like MS Word or WordPerfect.

Just as an example, here is a figure that I refer to as
Figure~\ref{figlabel}.

%\begin{figure}
%\centerline{\includegraphics[height=4in,angle=90]{proposal-pic}}
%\caption{Silly example of including a graphic file.}
%\label{figlabel}
%\end{figure}


\section{How and When}

Detail the  steps you plan  to take to  accomplish your goal  and what
resources you  will use.   Include a tentative  timeline.  Here  is an
example:

\begin{center}
\begin{tabular}{rrl}
1.  & March 5 & Read first paper\\
2.  & March 10 & Read second paper\\
3.  & end of March & Summarize both papers in about 5 pages\\
4.  & April 15 & Implement technique in first paper in Java\\
5.  & April 20 & Test the implementation with one example\\
6.  & April 30 & Write up my results in about 5 pages\\
7. & May 2 & Prepare oral report
\end{tabular}
\end{center}



\section{Grading}

I will evaluate your projects on how much effort I judge you have put
into it and how much you have learned.  Your grade will also depend on
the quality of your written report and of your oral presentation.  

It isn't required, but I strongly encourage you to hand in a draft of
your report a couple of weeks before it is due.  I will read them and
make written suggestions on how to improve it.  I also recommend you
discuss with me what you plan to present orally.  You will have 10 to
15 minutes only, and it is very difficult to put together an
understandable and interesting talk that is that short.

\section{References}

If you have found some papers already, list them here.

\end{document}




